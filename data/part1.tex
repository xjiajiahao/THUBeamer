\section{介绍}
\miniframesoff

%\subsection{课题背景和意义}
\frame
{
\frametitle{课题背景和意义}
\begin{block}{增量梯度方法}
  \begin{itemize}
    \item 机器学习领域中应用最广泛的一类优化方法
    \item 每次迭代只需计算少量的数据样本所对应的目标函数梯度
    \item 优点:迭代速度快、计算和存储复杂度较低、扩展性好等等
    \item 按抽取数据样本的方式可分为两类:
      \begin{itemize}
        \item 随机方法——随机梯度下降(SGD)及其变种
        \item \textbf{确定性方法}
      \end{itemize}
  \end{itemize}
\end{block}

\pause

\begin{block}{选题意义}
  \begin{itemize}
    \item 理论意义:有助于理解一般增量梯度方法的收敛性质
    \item 实践意义:有助于提高机器学习算法的实用性和可扩展性
  \end{itemize}
\end{block}
}

\frame
{
\frametitle{研究现状}
@TODO
}

\frame
{
\frametitle{本文工作}
\begin{itemize}
  \item 提出了“基于快照的增量梯度方法”(SIG)及其加速变种 \vspace*{1em}
  \item 证明了所提出的SIG方法达到线性收敛 \vspace*{1em}
  \item 将SIG算法推广到更一般的单调算子零点问题上,证明了在这一类问题上同样达到线性收敛
\end{itemize}
}

\miniframeson
