\section{理论结果}
\subsection{线性收敛结果}
  \frame
  {
    \frametitle{扰动分析框架}
    \footnotesize
    SIG内循环迭代:    $w_{t+1}^{s} = w_t^s - \eta \big(\nabla f_{i_t}(w_t^s) - \nabla f_{i_t}(w_0^s) + \frac{1}{N} \sum_{i=1}^{N} \nabla f_i(w_0^s) \big)$ \\~\\

\pause

    由于采用循环顺序访问数据样本,则对$t = 0, \ldots, N-1$求和得:
    $$
    \begin{aligned}
    w_0^{s+1} &= w_0^s - \eta\sum_{t=0}^{N-1} \nabla f_{t+1} (\alert<4->{w_t^s}) &\;\; \text{SIG} \\
\pause
    w_0^{s+1} &= w_0^s - \eta\sum_{i=1}^{N} \nabla f_{i} (\alert<4->{w_0^s}) &\;\; \text{梯度下降}
    \end{aligned}
    $$ \\~\\

  \uncover<5>
  {
    于是,可以将SIG当做\alert{受扰动的梯度下降}来分析
  }

  }

  \frame
  {
    \begin{block}{}
    \frametitle{重要引理:梯度误差上界}
    在光滑假设下,令SIG算法的步长$\eta$满足$\eta < \frac{1}{NL}$,那么
    $$
      \sum_{t=0}^{N-1} \| \nabla f_{i_t}(w_t^s)\! -\! \nabla f_{i_t}(w_0^s) \|
      \le \frac{\eta L^2 N^2}{2(1 - \eta L N)} \| w_0^s\! -\! w^* \|
    $$ \\~\\
    \end{block}

    \pause
    % \footnotesize
    这说明梯度估计量受到的扰动将随算法收敛而逐渐趋于0
  }


  \frame
  {
    \frametitle{定理1}
    \begin{block}{}
    在强凸和光滑假设下,令步长$\eta \le \mu/(2L^2N)$,则迭代参数到最优点的距离$\|w_0^s - w^*\|$线性收敛:
    $$
    \| w_0^{s+1} - w^* \| \le \alpha \| w_0^{s} - w^* \|,
    $$
    其中$\alpha = 1 - \frac{1}{2} \eta \mu N < 1$.
    \end{block}

\pause

    \footnotesize
    \begin{itemize}
        \item 定理1不仅适用于确定性的样本选择顺序,也适用于任意排列顺序
        \item SIG算法达到$\epsilon$精度的最坏复杂度(迭代次数)为$\mathcal{O}(N\kappa^2 \text{log}(1/\epsilon))$, \\
        而IAG算法的最坏复杂度为$\mathcal{O}(N^2 \kappa \text{log}(1/\epsilon))$
    \end{itemize}
  }

\subsection{推广:单调算子}
  \frame
  {
    \frametitle{单调算子的零点问题}
    \footnotesize
    SIG算法可推广到求解单调算子的零点问题:$0 \in B(w) = \frac{1}{n}\sum_{i=1}^n B_i(w)$ \\~\\

    \pause

    \textbf{假设:}
    \begin{itemize}
        \item 假设每个$B_i(\cdot)$均有Lipschitz常数$L > 0$,即
        $$
        \| B_i(w_1) - B_i(w_2) \|_2 \le L \| w_1 - w_2 \|_2 \;\; \forall w_1, w_2 \in \mathbb{R}^d
        $$
        \item $B(\cdot)$是$\mu$-强单调算子,即
        $$
        \langle B(w_1) - B(w_2), w_1 - w_2 \rangle \ge \mu \| w_1 - w_2 \|^2, \;\; \forall w_1, w_2 \in \mathbb{R}^d
        $$
    \end{itemize}

    \pause
    \begin{block}{}
    \textbf{应用:}鞍点问题,即$\underset{w \in \mathbb{R}^d}{\text{min}} \ \underset{u \in \mathbb{R}^p}{\text{max}} F(w, u)$
    \begin{itemize}
        \item $F$对第一个变量是凸函数,对第二个变量是凹函数
        \item 具体应用:分类模型的 ROC 曲线下面积最大化问题,图像处理中的去噪、去抖动问题
    \end{itemize}
    \end{block}
  }

  \frame
  {
    \frametitle{定理2}
    \begin{block}{}
    在Lipschitz和强单调假设下,令步长$\eta \le \mu/(2L^2N)$,则迭代参数到最优点的距离$\|w_0^s - w^*\|$线性收敛:
    $$
    \| w_0^{s+1} - w^* \| \le \beta \| w_0^{s} - w^* \|,
    $$
    其中$\beta = 1 - \frac{1}{4} \eta \mu N < 1$.
    \end{block}
  }
